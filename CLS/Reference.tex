%!TEX TS-program = xelatex
%!TEX encoding = UTF-8 Unicode

\section*{Preface}

Coral is a Ruby-like programming language which enhances advanced object- oriented programming with elements of functional programming. Every value is an object, in this sense it is a pure object-oriented language. Object blueprints are described by classes. Classes can be composed in multiple ways – classic inheritance, mixin composition, union and compound types.

Coral is also a functional language in the sense that every function is also an object. Therefore, function definitions can be nested and higher-order functions are supported out-of-the-box. Coral also has a limited support for pattern matching, which can emulate the algebraic types used in other functional languages.

Coral has been developed from 2012 in a home environment out of pure enthusiasm for programming and out of a desire for a truly versatile language. This document is a work in progress and will stay that way forever. It acts as a reference for the language definition and some core library classes.

\chapter{Lexical Syntax}

Coral programs are written using the Unicode character set; Unicode supplementary characters are supported as well. Coral programs are preferably encoded with the \textsc{UTF-8} character encoding. While every Unicode character is supported, usage of Unicode escapes is encouraged, since fonts that IDEs might use may not support the full Unicode character set.

\newpage

\section{Identifiers}\label{sec:identifiers}

\section{Keywords}\label{sec:keywords}

\section{Newline Characters}\label{sec:newlinecharacters}

\section{Operators}\label{sec:operators}

\section{Literals}\label{sec:literals}

\subsection{Integer Literals}\label{sec:integerliterals}

\subsection{Floating Point Literals}\label{sec:floatliterals}

\subsection{Boolean Literals}\label{sec:booleanliterals}

\subsection{String Literals}\label{sec:stringliterals}

\subsection{Symbol Literals}\label{sec:symbolliterals}

\subsection{Type Parameters}\label{sec:typeparameterliterals}

\subsection{Regular Expression Literals}\label{sec:regexpliterals}

\subsection{Collection Literals}\label{sec:collectionliterals}

\section{Whitespace \& Comments}\label{sec:whitespacecomments}

\section{Preprocessor Macros}\label{sec:preprocessormacros}

\chapter{Identifiers, Names \& Scopes}

\chapter{Types}

\section{Paths}
